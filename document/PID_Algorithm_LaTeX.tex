\documentclass{article}
\usepackage{algorithm}
\usepackage{algorithmic}
\usepackage{amsmath}
\usepackage{amssymb}

\begin{document}

\section{PID Controller with Anti-Derivative Kick and Intelligent Integral Windup Protection}

\subsection{Algorithm Description}

This PID controller implements an incremental structure with two key features: Anti-Derivative Kick to prevent output spikes during setpoint changes, and Intelligent Integral Windup Protection to prevent integral saturation.

\begin{algorithm}
\caption{PID Controller with Anti-Derivative Kick and Intelligent Integral Windup Protection}
\label{alg:pid_enhanced}
\begin{algorithmic}[1]
\REQUIRE $r(k)$: setpoint, $y(k)$: measured value, $K_p, K_i, K_d$: PID parameters
\ENSURE $u(k)$: controller output
\STATE Initialize: $e_{prev} \leftarrow 0$, $y_{prev} \leftarrow 0$, $u_{prev} \leftarrow 0$, $I \leftarrow 0$
\STATE Set limits: $u_{min}$, $u_{max}$
\WHILE{system is running}
    \STATE $e(k) \leftarrow r(k) - y(k)$ \COMMENT{Calculate current error}
    \STATE $P \leftarrow K_p \cdot e(k)$ \COMMENT{Proportional term}
    \STATE $I \leftarrow I + K_i \cdot e(k)$ \COMMENT{Integral term accumulation}
    \STATE $D \leftarrow K_d \cdot (y_{prev} - y(k))$ \COMMENT{Anti-derivative kick}
    \STATE $u(k) \leftarrow u_{prev} + P + I + D$ \COMMENT{Incremental PID output}
    \IF{$u(k) > u_{max}$}
        \STATE $I \leftarrow I - (u(k) - u_{max})$ \COMMENT{Intelligent integral windup protection}
        \STATE $u(k) \leftarrow u_{max}$
    \ELSIF{$u(k) < u_{min}$}
        \STATE $I \leftarrow I - (u(k) - u_{min})$ \COMMENT{Intelligent integral windup protection}
        \STATE $u(k) \leftarrow u_{min}$
    \ENDIF
    \STATE $e_{prev} \leftarrow e(k)$, $y_{prev} \leftarrow y(k)$, $u_{prev} \leftarrow u(k)$
    \STATE \textbf{output} $u(k)$
\ENDWHILE
\end{algorithmic}
\end{algorithm}

\subsection{Mathematical Formulation}

The discrete-time PID controller with the proposed enhancements can be expressed as:

\begin{equation}
u(k) = u(k-1) + K_p e(k) + K_i e(k) + K_d [y(k-1) - y(k)]
\end{equation}

where:
\begin{itemize}
    \item $u(k)$ is the controller output at sample $k$
    \item $e(k) = r(k) - y(k)$ is the error at sample $k$
    \item $r(k)$ and $y(k)$ are the setpoint and measured values respectively
    \item $K_p$, $K_i$, $K_d$ are the proportional, integral, and derivative gains
\end{itemize}

\subsection{Key Innovations}

\subsubsection{Anti-Derivative Kick}

Traditional derivative term: $D = K_d[e(k) - e(k-1)]$

Enhanced derivative term: $D = K_d[y(k-1) - y(k)]$

This modification prevents derivative spikes when the setpoint changes abruptly, as the derivative is computed based on the measured variable rather than the error.

\subsubsection{Intelligent Integral Windup Protection}

The integral windup protection mechanism adjusts the integral term before output limitation:

\begin{equation}
I_{adjusted} = I - [u(k) - u_{limit}]
\end{equation}

where $u_{limit}$ is either $u_{max}$ or $u_{min}$ depending on which limit is exceeded.

\subsection{Functional Programming Design Pattern}

The PID controller implementation employs functional programming design patterns through lambda expressions for sensor reading and output conversion functions, achieving complete decoupling between control algorithms and hardware interfaces:

\begin{algorithm}
\caption{Hardware Interface Binding through Function Pointers}
\label{alg:function_binding}
\begin{algorithmic}[1]
\REQUIRE PID controller instance, sensor interface, actuator interface
\ENSURE Configured PID controller with bound I/O functions
\STATE \textbf{Define sensor reading function:}
\STATE $\text{read\_sensor} \leftarrow \lambda() \rightarrow \text{double}$
\STATE \quad \textbf{return} $\text{INA226\_getCurrent\_mA()}$
\STATE
\STATE \textbf{Define output conversion function:}
\STATE $\text{convert\_output} \leftarrow \lambda(\text{output: double}) \rightarrow \text{double}$
\STATE \quad $\text{MCP4725\_setVoltage}(\text{output})$
\STATE \quad \textbf{return} $\text{output}$
\STATE
\STATE \textbf{Bind functions to PID controller:}
\STATE $\text{controller.read\_sensor} \leftarrow \text{read\_sensor}$
\STATE $\text{controller.convert\_output} \leftarrow \text{convert\_output}$
\end{algorithmic}
\end{algorithm}

This functional programming approach provides several advantages:
\begin{itemize}
    \item \textbf{Hardware abstraction}: Control logic is independent of specific sensor/actuator implementations
    \item \textbf{Enhanced testability}: Mock functions can be easily substituted for unit testing
    \item \textbf{Improved maintainability}: Hardware changes require only function binding modifications
    \item \textbf{Runtime flexibility}: I/O functions can be dynamically reconfigured during operation
\end{itemize}

\subsection{FreeRTOS Interrupt Management Paradigm}

The button handling system strictly follows FreeRTOS manual best practices for interrupt management, implementing a "Minimal ISR, Complex Task Processing" design paradigm. This approach ensures real-time performance while maintaining system responsiveness.

\begin{algorithm}
\caption{FreeRTOS Button Interrupt System Initialization}
\label{alg:interrupt_init}
\begin{algorithmic}[1]
\REQUIRE GPIO pins, interrupt priorities, queue and semaphore handles
\ENSURE Configured interrupt system with bound handlers
\STATE \textbf{System Initialization:}
\STATE Create message queue: $button\_queue \leftarrow \text{xQueueCreate}(queue\_size, item\_size)$
\STATE Create binary semaphore: $button\_semaphore \leftarrow \text{xSemaphoreCreateBinary}()$
\STATE
\FOR{each button pin $i$ in $button\_pins$}
    \STATE Configure GPIO: $\text{gpio\_config}(pin_i, \text{INPUT\_PULLUP})$
    \STATE Install ISR: $\text{gpio\_isr\_handler\_add}(pin_i, button\_isr\_handler, pin_i)$
    \STATE Set interrupt type: $\text{gpio\_set\_intr\_type}(pin_i, \text{FALLING\_EDGE})$
\ENDFOR
\STATE
\STATE Create button handler task: $\text{xTaskCreate}(button\_handler\_task, ...)$
\end{algorithmic}
\end{algorithm}

\begin{algorithm}
\caption{Minimal Interrupt Service Routine (ISR)}
\label{alg:minimal_isr}
\begin{algorithmic}[1]
\REQUIRE Triggered GPIO pin number
\ENSURE Message queued and task awakened
\STATE \textbf{function} $button\_isr\_handler(gpio\_pin)$
\STATE \quad $higher\_priority\_task\_woken \leftarrow \text{false}$
\STATE \quad \textbf{Send pin info to queue:}
\STATE \quad $\text{xQueueSendFromISR}(button\_queue, \&gpio\_pin, \&higher\_priority\_task\_woken)$
\STATE \quad \textbf{Release semaphore to wake task:}
\STATE \quad $\text{xSemaphoreGiveFromISR}(button\_semaphore, \&higher\_priority\_task\_woken)$
\STATE \quad \textbf{Yield if higher priority task awakened:}
\STATE \quad \textbf{if} $higher\_priority\_task\_woken$ \textbf{then}
\STATE \quad \quad $\text{portYIELD\_FROM\_ISR}(higher\_priority\_task\_woken)$
\STATE \quad \textbf{end if}
\STATE \textbf{end function}
\end{algorithmic}
\end{algorithm}

\begin{algorithm}
\caption{Complex Button Handler Task}
\label{alg:button_task}
\begin{algorithmic}[1]
\REQUIRE Initialized queues and semaphores
\ENSURE Complete button processing with debouncing and function mapping
\STATE \textbf{function} $button\_handler\_task()$
\WHILE{system running}
    \STATE \textbf{Wait for semaphore (blocking):}
    \STATE $\text{xSemaphoreTake}(button\_semaphore, \text{portMAX\_DELAY})$
    \STATE
    \STATE \textbf{Process all queued button events:}
    \WHILE{$\text{xQueueReceive}(button\_queue, \&gpio\_pin, 0) == \text{pdTRUE}$}
        \STATE \textbf{Debouncing:}
        \STATE $\text{vTaskDelay}(debounce\_time)$ \COMMENT{Hardware settling time}
        \STATE \textbf{if} $\text{gpio\_get\_level}(gpio\_pin) == \text{LOW}$ \textbf{then}
            \STATE $press\_start\_time \leftarrow \text{xTaskGetTickCount}()$
            \STATE \textbf{Wait for release and measure duration:}
            \WHILE{$\text{gpio\_get\_level}(gpio\_pin) == \text{LOW}$}
                \STATE $\text{vTaskDelay}(poll\_interval)$
            \ENDWHILE
            \STATE $press\_duration \leftarrow \text{xTaskGetTickCount}() - press\_start\_time$
            \STATE
            \STATE \textbf{Button type recognition and function mapping:}
            \IF{$press\_duration < short\_press\_threshold$}
                \STATE $\text{execute\_short\_press\_function}(gpio\_pin)$
            \ELSIF{$press\_duration < long\_press\_threshold$}
                \STATE $\text{execute\_long\_press\_function}(gpio\_pin)$
            \ELSE
                \STATE $\text{execute\_extra\_long\_press\_function}(gpio\_pin)$
            \ENDIF
        \STATE \textbf{end if}
    \ENDWHILE
\ENDWHILE
\STATE \textbf{end function}
\end{algorithmic}
\end{algorithm}

\subsubsection{Design Paradigm Advantages}

This FreeRTOS interrupt management approach provides several critical benefits:

\begin{itemize}
    \item \textbf{Minimal ISR execution time}: ISR completes within microseconds, maintaining real-time guarantees
    \item \textbf{Deterministic response}: Interrupt latency is predictable and bounded
    \item \textbf{Complex processing isolation}: Button logic runs in task context with full RTOS services available
    \item \textbf{Resource protection}: Queue and semaphore provide thread-safe communication between ISR and task
    \item \textbf{Power efficiency}: Task can be blocked, allowing CPU to enter low-power states
    \item \textbf{Scalability}: Additional buttons require only new ISR bindings, no architectural changes
\end{itemize}

\subsubsection{Timing Analysis}

\begin{equation}
T_{ISR} = T_{queue\_send} + T_{semaphore\_give} + T_{context\_switch}
\end{equation}

\begin{equation}
T_{total\_response} = T_{ISR} + T_{task\_scheduling} + T_{button\_processing}
\end{equation}

where $T_{ISR} < 10\mu s$ ensures minimal impact on real-time performance.

\subsection{Performance Characteristics}

\begin{itemize}
    \item \textbf{Reduced settling time}: Intelligent integral protection prevents overshoot recovery delays
    \item \textbf{Improved transient response}: Anti-derivative kick eliminates output spikes during setpoint changes
    \item \textbf{Enhanced stability}: Incremental structure provides better numerical stability
    \item \textbf{Robustness}: Suitable for systems with actuator limitations and frequent setpoint changes
\end{itemize}

\section{Data Acquisition System with Sensor Management}

\subsection{INA226 High-Precision Current/Voltage Sensor Task}

The data acquisition system implements a real-time sensor management paradigm with optimized I2C communication, overvoltage protection, and multi-channel data distribution for GUI systems.

\begin{algorithm}
\caption{INA226 Sensor Data Acquisition Task}
\label{alg:ina226_sensor}
\begin{algorithmic}[1]
\REQUIRE INA226 sensor with 20mΩ shunt, calibration coefficients
\ENSURE Real-time sensor data for control system
\STATE Initialize: I2C communication, sensor configuration
\STATE Configure: 8.3ms conversion time for optimal precision
\STATE Set calibration: $I_{cal\_A}$, $I_{cal\_B}$, $V_{cal\_A}$, $V_{cal\_B}$
\STATE Initialize message queues for multi-channel data distribution
\WHILE{task is active}
    \STATE Read raw sensor data from INA226 registers
    \STATE $I_{raw} \leftarrow$ INA226.getCurrent\_mA()
    \STATE $V_{raw} \leftarrow$ INA226.getBusVoltage()
    \STATE \COMMENT{Apply calibration correction}
    \STATE $I_{actual} \leftarrow I_{cal\_A} \times I_{raw} + I_{cal\_B}$
    \STATE $V_{actual} \leftarrow V_{cal\_A} \times V_{raw} + V_{cal\_B}$
    \STATE $P_{actual} \leftarrow |I_{actual} \times V_{actual}| / 1000$
    \STATE $R_{actual} \leftarrow |V_{actual}| / |I_{actual}| \times 1000$
    \STATE \COMMENT{Create data packets for different subsystems}
    \STATE Create queue elements for current, voltage, power, resistance
    \STATE \COMMENT{Overvoltage protection check}
    \IF{$V_{actual} \geq V_{warning}$}
        \STATE Trigger overvoltage protection semaphore
        \STATE Set protection flag and disable PID control
        \STATE Update GUI with warning status
    \ELSE
        \STATE Clear protection flags
        \STATE Resume normal operation
    \ENDIF
    \STATE Send data to GUI message queue with non-blocking send
    \STATE Wait for sampling interval (200ms)
\ENDWHILE
\end{algorithmic}
\end{algorithm}

\subsection{Overvoltage Protection Task}

The overvoltage protection system provides real-time safety monitoring with semaphore-triggered emergency responses.

\begin{algorithm}
\caption{Overvoltage Protection Task}
\label{alg:overvoltage_protection}
\begin{algorithmic}[1]
\REQUIRE Protection semaphore, emergency shutdown procedures
\ENSURE System safety under overvoltage conditions
\STATE Initialize protection semaphore as binary semaphore
\STATE Set emergency thresholds and response procedures
\WHILE{system is running}
    \STATE Wait for protection semaphore (blocking wait)
    \STATE \COMMENT{Emergency response triggered by sensor task}
    \STATE Disable PID controller output immediately
    \STATE Set DAC output to minimum safe level
    \STATE Update system status flags
    \STATE Log protection event with timestamp
    \STATE \COMMENT{Notify GUI system of protection status}
    \STATE Send emergency status to GUI queue
    \STATE \COMMENT{Implement hysteresis for protection recovery}
    \STATE Wait for voltage to drop below recovery threshold
    \STATE Gradually re-enable system after safety delay
\ENDWHILE
\end{algorithmic}
\end{algorithm}

\subsection{Encoder Input Processing System}

The encoder processing system supports both quadrature and single-edge modes with configurable range limiting for current setpoint adjustment.

\begin{algorithm}
\caption{Encoder Input Processing with Range Limiting}
\label{alg:encoder_processing}
\begin{algorithmic}[1]
\REQUIRE Rotary encoder with quadrature/single mode capability
\ENSURE Current setpoint within safe operating range [50mA, 1800mA]
\STATE Initialize encoder in QUAD mode (4x resolution)
\STATE Set range limits: $I_{min} = 50$mA, $I_{max} = 1800$mA
\STATE Initialize accumulator: $count_{total} \leftarrow 0$
\WHILE{encoder task is active}
    \STATE $count_{delta} \leftarrow$ encoder.read\_count\_accum\_clear()
    \STATE \COMMENT{Mode-dependent scaling}
    \IF{encoder.mode == QUAD}
        \STATE $count_{scaled} \leftarrow count_{delta} \times 10.0$ \COMMENT{Higher sensitivity}
    \ELSE \COMMENT{SINGLE mode}
        \STATE $count_{scaled} \leftarrow count_{delta}$ \COMMENT{Standard sensitivity}
    \ENDIF
    \STATE $count_{total} \leftarrow count_{total} + count_{scaled}$
    \STATE \COMMENT{Apply range limiting with clamping}
    \IF{$count_{total} > I_{max}$}
        \STATE $count_{total} \leftarrow I_{max}$
    \ELSIF{$count_{total} < I_{min}$}
        \STATE $count_{total} \leftarrow I_{min}$
    \ENDIF
    \STATE \COMMENT{Update control system only when not in protection mode}
    \IF{testing\_flag == FALSE AND overvoltage\_flag == FALSE}
        \STATE Update PID controller setpoint: $r(k) \leftarrow count_{total}$
        \STATE Convert to DAC voltage for direct control mode
        \STATE Send setpoint to GUI queue for display update
    \ENDIF
    \STATE Wait for encoder sampling interval
\ENDWHILE
\end{algorithmic}
\end{algorithm}

\subsection{Multi-Channel Data Distribution Architecture}

\begin{algorithm}
\caption{Message Queue Data Distribution System}
\label{alg:data_distribution}
\begin{algorithmic}[1]
\REQUIRE Multiple sensor tasks, GUI system, control loops
\ENSURE Efficient real-time data distribution
\STATE Initialize message queues for each data type
\STATE Define queue element structure with task ID and data type
\WHILE{system is operational}
    \STATE \COMMENT{Sensor tasks populate queues}
    \FORALL{sensor tasks}
        \STATE Create typed queue element with source identification
        \STATE Attempt non-blocking queue send to prevent task blocking
        \IF{queue full}
            \STATE Log overflow condition
            \STATE Implement priority-based queue management
        \ENDIF
    \ENDFOR
    \STATE \COMMENT{Consumer tasks process data}
    \FORALL{consumer tasks (GUI, Control, Logging)}
        \STATE Attempt queue receive with timeout
        \STATE Process data according to consumer type
        \STATE Update respective subsystem state
    \ENDFOR
\ENDWHILE
\end{algorithmic}
\end{algorithm}

\subsection{Mathematical Analysis of Sensor System}

The sensor calibration follows a linear correction model:

\begin{equation}
I_{corrected} = A_I \cdot I_{measured} + B_I
\end{equation}

\begin{equation}
V_{corrected} = A_V \cdot V_{measured} + B_V
\end{equation}

where $A_I, B_I, A_V, B_V$ are calibration coefficients determined through precision measurement.

The encoder resolution in quadrature mode provides:

\begin{equation}
Resolution_{current} = \frac{I_{max} - I_{min}}{Encoder_{steps}} = \frac{1800 - 50}{2^{16}} \approx 0.027 \text{mA/step}
\end{equation}

\subsection{Performance Characteristics}

\begin{itemize}
    \item \textbf{Sensor precision}: 16-bit ADC with 8.3ms conversion time for optimal noise rejection
    \item \textbf{Calibration accuracy}: Linear correction achieves <1\% measurement error
    \item \textbf{Protection response time}: Semaphore-based protection provides <10ms emergency response
    \item \textbf{Encoder resolution}: Quadrature mode enables 0.027mA current adjustment resolution
    \item \textbf{Data throughput}: Multi-queue architecture supports concurrent data streams without blocking
    \item \textbf{Range safety}: Hardware-enforced current limiting prevents equipment damage
\end{itemize}

\subsection{Data Acquisition System Pseudocode}

\subsubsection{System Overview}

The data acquisition system is centered around the INA226 high-precision current/voltage sensor, communicating via I2C bus. The system employs a FreeRTOS multi-task architecture to achieve parallel execution of sensor data acquisition, overvoltage protection, and encoder input processing.

\begin{algorithm}
\caption{Generic Sensor Data Acquisition Task Framework}
\label{alg:sensor_data_acquisition}
\begin{algorithmic}[1]
\REQUIRE Initialized sensor interface, queue handles, semaphore handles
\ENSURE Continuous data acquisition and real-time safety monitoring
\STATE \textbf{function} $sensor\_data\_acquisition\_task()$
\STATE
\STATE \textbf{Sensor initialization configuration:}
\STATE Configure I2C bus parameters
\STATE Set sensor operating mode \COMMENT{e.g., INA226 shunt resistor value 20mΩ}
\STATE Set conversion time parameters \COMMENT{Current and voltage conversion time both 8.3ms}
\STATE Execute sensor calibration procedure
\STATE
\WHILE{System running}
    \STATE \textbf{Data acquisition phase:}
    \STATE $voltage \leftarrow \text{read\_sensor\_voltage}()$
    \STATE $current \leftarrow \text{read\_sensor\_current}()$
    \STATE $power \leftarrow \text{calculate\_power}(voltage, current)$
    \STATE $resistance \leftarrow \text{calculate\_resistance}(voltage, current)$
    \STATE
    \STATE \textbf{Data calibration processing:}
    \STATE $voltage_{calibrated} \leftarrow A_{voltage} \times voltage + B_{voltage}$
    \STATE $current_{calibrated} \leftarrow A_{current} \times current + B_{current}$
    \STATE
    \STATE \textbf{Safety monitoring check:}
    \IF{$voltage_{calibrated} > WARNING\_VOLTAGE$}
        \STATE $\text{xSemaphoreGive}(overvoltage\_protection\_semaphore)$ \COMMENT{Trigger overvoltage protection}
        \STATE Log overvoltage event to logging system
    \ENDIF
    \STATE
    \STATE \textbf{Data distribution mechanism:}
    \STATE Construct data packet: $data\_packet \leftarrow \{type, value, timestamp\}$
    \FOR{Each data type $data\_type$ in $\{voltage, current, power, resistance\}$}
        \STATE $packet.type \leftarrow data\_type\_identifier$
        \STATE $packet.value \leftarrow calibrated\_value$
        \STATE $packet.timestamp \leftarrow \text{xTaskGetTickCount}()$
        \STATE $\text{xQueueSend}(gui\_data\_queue, \&packet, queue\_timeout)$
    \ENDFOR
    \STATE
    \STATE $\text{vTaskDelay}(sampling\_interval)$ \COMMENT{200ms sampling interval}
\ENDWHILE
\STATE \textbf{end function}
\end{algorithmic}
\end{algorithm}

\begin{algorithm}
\caption{Overvoltage Protection Task}
\label{alg:overvoltage_protection}
\begin{algorithmic}[1]
\REQUIRE Overvoltage protection semaphore, system control interface
\ENSURE Real-time overvoltage protection response
\STATE \textbf{function} $overvoltage\_protection\_task()$
\WHILE{System running}
    \STATE \textbf{Wait for overvoltage signal:}
    \STATE $\text{xSemaphoreTake}(overvoltage\_protection\_semaphore, \text{portMAX\_DELAY})$
    \STATE
    \STATE \textbf{Emergency protection measures:}
    \STATE Immediately disconnect load path \COMMENT{Hardware safety measure}
    \STATE Stop PID controller output
    \STATE Set DAC output to safe value
    \STATE
    \STATE \textbf{System status handling:}
    \STATE Update system protection status flag
    \STATE Send warning message to GUI
    \STATE Log detailed protection event information
    \STATE
    \STATE \textbf{Recovery condition check:}
    \WHILE{Protection status active}
        \STATE Read current voltage value
        \IF{$voltage < SAFE\_VOLTAGE\_THRESHOLD$}
            \STATE Clear protection status
            \STATE Allow system restart
            \STATE \textbf{break}
        \ENDIF
        \STATE $\text{vTaskDelay}(protection\_check\_interval)$
    \ENDWHILE
\ENDWHILE
\STATE \textbf{end function}
\end{algorithmic}
\end{algorithm}

\begin{algorithm}
\caption{Encoder Input Processing Task}
\label{alg:encoder_input_processing}
\begin{algorithmic}[1]
\REQUIRE Encoder hardware interface, PID controller handle
\ENSURE Real-time mapping from user input to control setpoint
\STATE \textbf{function} $encoder\_input\_processing\_task()$
\STATE
\STATE \textbf{Encoder initialization:}
\STATE Configure encoder operating mode \COMMENT{QUAD quadrature or SINGLE mode}
\STATE Set encoder count range and resolution
\STATE Clear accumulator counter
\STATE
\WHILE{System running}
    \STATE \textbf{Read encoder data:}
    \STATE $raw\_count \leftarrow \text{encoder.read\_count\_accum\_clear}()$
    \STATE
    \STATE \textbf{Data processing and range limiting:}
    \STATE $current\_setpoint \leftarrow previous\_setpoint + raw\_count \times scale\_factor$
    \STATE
    \STATE \textbf{Apply system constraints:}    \IF{$current\_setpoint < MIN\_CURRENT\_LIMIT$} \COMMENT{50mA lower limit}
        \STATE $current\_setpoint \leftarrow MIN\_CURRENT\_LIMIT$
    \ELSIF{$current\_setpoint > MAX\_CURRENT\_LIMIT$} \COMMENT{1800mA upper limit}
        \STATE $current\_setpoint \leftarrow MAX\_CURRENT\_LIMIT$
    \ENDIF
    \STATE    \STATE \textbf{Update control system:}
    \STATE $\text{pid\_controller.set\_setpoint}(current\_setpoint)$
    \STATE Update GUI display value
    \STATE    \STATE \textbf{User feedback:}
    \IF{Setpoint value changed}
        \STATE Trigger interface update event
        \STATE Optional: Provide audio/haptic feedback
    \ENDIF
    \STATE
    \STATE $\text{vTaskDelay}(encoder\_polling\_interval)$ \COMMENT{Fast response polling interval}
\ENDWHILE
\STATE \textbf{end function}
\end{algorithmic}
\end{algorithm}

\subsubsection{Data Acquisition System Architecture Characteristics}

\begin{itemize}
    \item \textbf{Multi-task parallel architecture}: Based on FreeRTOS to implement independent task processing for sensor acquisition, safety protection, and user input
    \item \textbf{Real-time safety monitoring}: Data acquisition is tightly coupled with overvoltage protection to ensure system safety
    \item \textbf{High-precision measurement}: Optimized INA226 sensor configuration balancing measurement accuracy and response speed
    \item \textbf{Queued data communication}: Thread-safe data transmission between tasks through message queues
    \item \textbf{User interaction response}: Encoder input directly maps to PID control setpoint, providing intuitive user experience
\end{itemize}

\subsubsection{System Performance Parameters}

\begin{equation}
T_{sampling} = T_{conversion} + T_{calculation} + T_{communication} = 8.3ms + <1ms + <1ms
\end{equation}

\begin{equation}
Response_{encoder} = T_{polling} + T_{processing} + T_{pid\_update} < 10ms
\end{equation}

where the total response time of the data acquisition system is approximately 200ms (limited by sampling interval), while the encoder input response time is within 10ms, ensuring good user interaction experience.

\end{document}
