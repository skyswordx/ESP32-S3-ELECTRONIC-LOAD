\documentclass{article}
\usepackage{algorithm}
\usepackage{algorithmic}
\usepackage{amsmath}
\usepackage{amssymb}

\begin{document}

\section{PID Controller Lambda Binding Design Pattern}

\subsection{Functional Programming Design Pattern}

The PID controller implementation employs functional programming design patterns through lambda expressions for sensor reading and output conversion functions, achieving complete decoupling between control algorithms and hardware interfaces.

\begin{algorithm}
\caption{Hardware Interface Binding through Function Pointers}
\label{alg:function_binding}
\begin{algorithmic}[1]
\REQUIRE PID controller instance, sensor interface, actuator interface
\ENSURE Configured PID controller with bound I/O functions
\STATE \textbf{Define sensor reading function:}
\STATE $\text{read\_sensor} \leftarrow \lambda() \rightarrow \text{double}$
\STATE \quad \textbf{return} $\text{INA226\_getCurrent\_mA()}$
\STATE
\STATE \textbf{Define output conversion function:}
\STATE $\text{convert\_output} \leftarrow \lambda(\text{output: double}) \rightarrow \text{double}$
\STATE \quad $\text{MCP4725\_setVoltage}(\text{output})$
\STATE \quad \textbf{return} $\text{output}$
\STATE
\STATE \textbf{Bind functions to PID controller:}
\STATE $\text{controller.read\_sensor} \leftarrow \text{read\_sensor}$
\STATE $\text{controller.convert\_output} \leftarrow \text{convert\_output}$
\end{algorithmic}
\end{algorithm}

\subsection{Architecture Advantages}

This functional programming approach provides several advantages:
\begin{itemize}
    \item \textbf{Hardware abstraction}: Control logic is independent of specific sensor/actuator implementations
    \item \textbf{Enhanced testability}: Mock functions can be easily substituted for unit testing
    \item \textbf{Improved maintainability}: Hardware changes require only function binding modifications
    \item \textbf{Runtime flexibility}: I/O functions can be dynamically reconfigured during operation
\end{itemize}

\subsection{Implementation Example}

\begin{algorithm}
\caption{Lambda Function Implementation for Sensor Interface}
\label{alg:lambda_implementation}
\begin{algorithmic}[1]
\REQUIRE Sensor hardware interface, data processing requirements
\ENSURE Bound lambda functions for PID controller
\STATE \textbf{Current sensor reading lambda:}
\STATE $\text{current\_reader} \leftarrow \lambda() \{$
\STATE \quad $raw\_value \leftarrow \text{INA226.getCurrent\_raw()}$
\STATE \quad $calibrated\_value \leftarrow raw\_value \times cal\_factor + offset$
\STATE \quad \textbf{return} $calibrated\_value$
\STATE $\}$
\STATE
\STATE \textbf{Voltage output lambda:}
\STATE $\text{voltage\_writer} \leftarrow \lambda(output\_voltage) \{$
\STATE \quad $dac\_value \leftarrow (output\_voltage / V_{ref}) \times 4095$
\STATE \quad $\text{MCP4725.setVoltage}(dac\_value)$
\STATE \quad \textbf{return} $output\_voltage$
\STATE $\}$
\STATE
\STATE \textbf{Controller binding:}
\STATE $\text{pid\_controller.bind\_input}(current\_reader)$
\STATE $\text{pid\_controller.bind\_output}(voltage\_writer)$
\end{algorithmic}
\end{algorithm}

\subsection{Design Pattern Benefits}

\subsubsection{Decoupling and Modularity}

The lambda binding pattern creates a clear separation between:
\begin{itemize}
    \item Control algorithm logic (PID computation)
    \item Hardware interface details (sensor/actuator communication)
    \item Data processing operations (calibration, scaling)
\end{itemize}

\subsubsection{Testing and Validation}

Mock functions can be easily substituted for hardware interfaces:

\begin{algorithm}
\caption{Mock Function Binding for Testing}
\label{alg:mock_binding}
\begin{algorithmic}[1]
\REQUIRE Test scenarios, simulation parameters
\ENSURE Testable PID controller configuration
\STATE \textbf{Mock sensor function:}
\STATE $\text{mock\_sensor} \leftarrow \lambda() \{$
\STATE \quad \textbf{return} $test\_value + noise\_generator()$
\STATE $\}$
\STATE
\STATE \textbf{Mock actuator function:}
\STATE $\text{mock\_output} \leftarrow \lambda(value) \{$
\STATE \quad $test\_log.append(value)$
\STATE \quad \textbf{return} $value$
\STATE $\}$
\STATE
\STATE \textbf{Test configuration:}
\STATE $\text{pid\_controller.bind\_input}(mock\_sensor)$
\STATE $\text{pid\_controller.bind\_output}(mock\_output)$
\end{algorithmic}
\end{algorithm}

\end{document}
