\documentclass{article}
\usepackage{algorithm}
\usepackage{algorithmic}
\usepackage{amsmath}
\usepackage{amssymb}

\begin{document}

\section{PID Controller with Anti-Derivative Kick and Intelligent Integral Windup Protection}

\subsection{Algorithm Description}

This PID controller implements an incremental structure with two key features: Anti-Derivative Kick to prevent output spikes during setpoint changes, and Intelligent Integral Windup Protection to prevent integral saturation.

\begin{algorithm}
\caption{PID Controller with Anti-Derivative Kick and Intelligent Integral Windup Protection}
\label{alg:pid_enhanced}
\begin{algorithmic}[1]
\REQUIRE $r(k)$: setpoint, $y(k)$: measured value, $K_p, K_i, K_d$: PID parameters
\ENSURE $u(k)$: controller output
\STATE Initialize: $e_{prev} \leftarrow 0$, $y_{prev} \leftarrow 0$, $u_{prev} \leftarrow 0$, $I \leftarrow 0$
\STATE Set limits: $u_{min}$, $u_{max}$
\WHILE{system is running}
    \STATE $e(k) \leftarrow r(k) - y(k)$ \COMMENT{Calculate current error}
    \STATE $P \leftarrow K_p \cdot e(k)$ \COMMENT{Proportional term}
    \STATE $I \leftarrow I + K_i \cdot e(k)$ \COMMENT{Integral term accumulation}
    \STATE $D \leftarrow K_d \cdot (y_{prev} - y(k))$ \COMMENT{Anti-derivative kick}
    \STATE $u(k) \leftarrow u_{prev} + P + I + D$ \COMMENT{Incremental PID output}
    \IF{$u(k) > u_{max}$}
        \STATE $I \leftarrow I - (u(k) - u_{max})$ \COMMENT{Intelligent integral windup protection}
        \STATE $u(k) \leftarrow u_{max}$
    \ELSIF{$u(k) < u_{min}$}
        \STATE $I \leftarrow I - (u(k) - u_{min})$ \COMMENT{Intelligent integral windup protection}
        \STATE $u(k) \leftarrow u_{min}$
    \ENDIF
    \STATE $e_{prev} \leftarrow e(k)$, $y_{prev} \leftarrow y(k)$, $u_{prev} \leftarrow u(k)$
    \STATE \textbf{output} $u(k)$
\ENDWHILE
\end{algorithmic}
\end{algorithm}

\subsection{Mathematical Formulation}

The discrete-time PID controller with the proposed enhancements can be expressed as:

\begin{equation}
u(k) = u(k-1) + K_p e(k) + K_i e(k) + K_d [y(k-1) - y(k)]
\end{equation}

where:
\begin{itemize}
    \item $u(k)$ is the controller output at sample $k$
    \item $e(k) = r(k) - y(k)$ is the error at sample $k$
    \item $r(k)$ and $y(k)$ are the setpoint and measured values respectively
    \item $K_p$, $K_i$, $K_d$ are the proportional, integral, and derivative gains
\end{itemize}

\subsection{Key Innovations}

\subsubsection{Anti-Derivative Kick}

Traditional derivative term: $D = K_d[e(k) - e(k-1)]$

Enhanced derivative term: $D = K_d[y(k-1) - y(k)]$

This modification prevents derivative spikes when the setpoint changes abruptly, as the derivative is computed based on the measured variable rather than the error.

\subsubsection{Intelligent Integral Windup Protection}

The integral windup protection mechanism adjusts the integral term before output limitation:

\begin{equation}
I_{adjusted} = I - [u(k) - u_{limit}]
\end{equation}

where $u_{limit}$ is either $u_{max}$ or $u_{min}$ depending on which limit is exceeded.

\subsection{Performance Characteristics}

\begin{itemize}
    \item \textbf{Reduced settling time}: Intelligent integral protection prevents overshoot recovery delays
    \item \textbf{Improved transient response}: Anti-derivative kick eliminates output spikes during setpoint changes
    \item \textbf{Enhanced stability}: Incremental structure provides better numerical stability
    \item \textbf{Robustness}: Suitable for systems with actuator limitations and frequent setpoint changes
\end{itemize}

\end{document}
