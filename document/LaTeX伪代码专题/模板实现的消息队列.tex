\documentclass{article}
\usepackage{algorithm}
\usepackage{algorithmic}
\usepackage{amsmath}
\usepackage{amssymb}

\begin{document}

\section{Template-Based Message Queue Implementation}

\subsection{Overview}

The template-based message queue implementation provides a type-safe, reusable communication infrastructure for FreeRTOS multi-task systems. The design employs C++ templates to achieve compile-time type checking while maintaining runtime efficiency.

\begin{algorithm}
\caption{Template Message Queue Element Structure}
\label{alg:queue_element_template}
\begin{algorithmic}[1]
\REQUIRE Template parameter T for data type
\ENSURE Type-safe queue element with task identification and data description
\STATE \textbf{Template Class Definition:}
\STATE $\text{QueueElement\_t<T>}$ \COMMENT{Generic queue element}
\STATE \quad $task\_id: task\_id\_t$ \COMMENT{Task identifier with priority encoding}
\STATE \quad $data: T$ \COMMENT{Templated data payload}
\STATE \quad $data\_description: data\_description\_t$ \COMMENT{Data type descriptor}
\STATE
\STATE \textbf{Constructor Overloads:}
\STATE $\text{QueueElement\_t}()$ \COMMENT{Default constructor}
\STATE \quad $task\_id \leftarrow TASK\_UNKNOWN$
\STATE \quad $data \leftarrow T()$
\STATE \quad $data\_description \leftarrow DATA\_DESCRIPTION\_NONE$
\STATE
\STATE $\text{QueueElement\_t}(task\_id\_t\ id, data\_description\_t\ desc, T\ value)$
\STATE \quad $task\_id \leftarrow id$
\STATE \quad $data \leftarrow value$
\STATE \quad $data\_description \leftarrow desc$
\end{algorithmic}
\end{algorithm}

\begin{algorithm}
\caption{Priority-Based Task Identification System}
\label{alg:priority_task_enum}
\begin{algorithmic}[1]
\REQUIRE System tasks with different priority levels
\ENSURE Enumerated task IDs with embedded priority information
\STATE \textbf{Priority-Encoded Task Enumeration:}
\STATE $EVENT\_OVER\_VOLTAGE \leftarrow 0$ \COMMENT{Highest priority: Emergency events}
\STATE $TASK\_ENCODER \leftarrow 1$ \COMMENT{High priority: User input}
\STATE $TASK\_DUMMY\_SENSOR \leftarrow 2$ \COMMENT{Medium priority: Mock data}
\STATE $TASK\_INA226 \leftarrow 3$ \COMMENT{Medium priority: Real sensor data}
\STATE $TASK\_ADC1 \leftarrow 4$ \COMMENT{Low priority: Auxiliary measurements}
\STATE $EVENT\_TESTING\_LOAD\_RATE \leftarrow 5$ \COMMENT{Low priority: Test functions}
\STATE $TASK\_UNKNOWN \leftarrow 99$ \COMMENT{Lowest priority: Unclassified}
\STATE
\STATE \textbf{Data Type Classification:}
\STATE $DATA\_DESCRIPTION\_VOLTAGE \leftarrow 1$
\STATE $DATA\_DESCRIPTION\_CURRENT \leftarrow 2$
\STATE $DATA\_DESCRIPTION\_POWER \leftarrow 3$
\STATE $DATA\_DESCRIPTION\_RESISTANCE \leftarrow 4$
\STATE $DATA\_DESCRIPTION\_CURRENT\_SETPOINT \leftarrow 6$
\STATE $DATA\_DESCRIPTION\_EMERGENCY\_STOP \leftarrow 7$
\end{algorithmic}
\end{algorithm}

\begin{algorithm}
\caption{Multi-Queue Communication Architecture}
\label{alg:multi_queue_system}
\begin{algorithmic}[1]
\REQUIRE Multiple FreeRTOS tasks requiring different communication patterns
\ENSURE Efficient message routing with minimal coupling
\STATE \textbf{Queue Initialization:}
\STATE $LVGL\_queue \leftarrow \text{xQueueCreate}(1000, \text{sizeof}(\text{QueueElement\_t<double>}))$
\STATE $current\_control\_queue \leftarrow \text{xQueueCreate}(10, \text{sizeof}(\text{QueueElement\_t<double>}))$
\STATE $button\_queue \leftarrow \text{xQueueCreate}(2, \text{sizeof}(\text{gpio\_num\_t}))$
\STATE
\STATE \textbf{Producer Task Pattern:}
\STATE \textbf{function} $sensor\_data\_producer()$
\WHILE{task active}
    \STATE $element \leftarrow \text{QueueElement\_t<double>}(TASK\_INA226, DATA\_DESCRIPTION\_CURRENT, current\_value)$
    \STATE $result \leftarrow \text{xQueueSend}(LVGL\_queue, \&element, 0)$
    \IF{$result == \text{errQUEUE\_FULL}$}
        \STATE Log queue overflow condition
    \ENDIF
    \STATE $\text{vTaskDelay}(sampling\_interval)$
\ENDWHILE
\STATE \textbf{end function}
\end{algorithmic}
\end{algorithm}

\begin{algorithm}
\caption{Consumer Task with Type-Safe Processing}
\label{alg:consumer_task_pattern}
\begin{algorithmic}[1]
\REQUIRE Initialized message queues and processing functions
\ENSURE Type-safe message consumption with routing
\STATE \textbf{function} $gui\_update\_consumer()$
\STATE $received\_element: \text{QueueElement\_t<double>}$
\WHILE{task active}
    \STATE $result \leftarrow \text{xQueueReceive}(LVGL\_queue, \&received\_element, \text{portMAX\_DELAY})$
    \IF{$result == \text{pdTRUE}$}
        \STATE \textbf{Route by task ID and data description:}
        \IF{$received\_element.task\_id == TASK\_INA226$}
            \IF{$received\_element.data\_description == DATA\_DESCRIPTION\_CURRENT$}
                \STATE $\text{update\_current\_display}(received\_element.data)$
            \ELSIF{$received\_element.data\_description == DATA\_DESCRIPTION\_VOLTAGE$}
                \STATE $\text{update\_voltage\_display}(received\_element.data)$
            \ELSIF{$received\_element.data\_description == DATA\_DESCRIPTION\_POWER$}
                \STATE $\text{update\_power\_display}(received\_element.data)$
            \ENDIF
        \ELSIF{$received\_element.task\_id == TASK\_ADC1$}
            \STATE $\text{update\_temperature\_display}(received\_element.data)$
        \ENDIF
    \ENDIF
\ENDWHILE
\STATE \textbf{end function}
\end{algorithmic}
\end{algorithm}

\subsection{Design Advantages}

\subsubsection{Type Safety and Compile-Time Verification}

The template-based approach provides several critical benefits:

\begin{itemize}
    \item \textbf{Compile-time type checking}: Template instantiation ensures data type consistency
    \item \textbf{Zero runtime type overhead}: No runtime type information required
    \item \textbf{Generic reusability}: Same queue structure supports different data types
    \item \textbf{Memory efficiency}: Optimized memory layout for each instantiated type
\end{itemize}

\subsubsection{Priority-Embedded Task Identification}

The task ID enumeration embeds priority information directly:

\begin{equation}
Priority_{effective} = \frac{1}{task\_id + 1}
\end{equation}

where lower numeric values correspond to higher priority levels.

\subsubsection{Scalable Communication Patterns}

\begin{itemize}
    \item \textbf{One-to-many broadcasting}: Single producer, multiple consumers
    \item \textbf{Many-to-one aggregation}: Multiple producers, single consumer  
    \item \textbf{Priority-based message handling}: Emergency messages bypass normal queuing
    \item \textbf{Type-specific routing}: Data description enables precise message routing
\end{itemize}

\subsection{Performance Characteristics}

\begin{equation}
T_{enqueue} = O(1) + T_{mutex} + T_{notification}
\end{equation}

\begin{equation}
T_{dequeue} = O(1) + T_{context\_switch} + T_{processing}
\end{equation}

\begin{equation}
Memory_{queue} = queue\_size \times sizeof(QueueElement\_t<T>) + overhead
\end{equation}

The template-based message queue system achieves predictable O(1) performance for both enqueue and dequeue operations, with memory usage scaling linearly with queue size and element size.

\end{document}
